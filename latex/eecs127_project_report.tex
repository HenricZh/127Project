\documentclass{amsart}

\usepackage{amsmath}
\usepackage{amssymb}
\usepackage{amsthm}

\usepackage{biblatex}
\addbibresource{references.bib}

\usepackage[margin=1in]{geometry}
\usepackage{float}

\usepackage{graphicx}
\usepackage{hyperref} 

% Feel free to add any of your own macros, etc here

\title{EECS 127 Project}  % Feel free to make more descriptive
\author{Author 1}
\author{Author 2}  % Feel free to delete if unnecessary.
\author{Author 3}  % Feel free to delete if unnecessary.

\begin{document}

\maketitle

% May want \tableofcontents if you submit a long report with sections
% \section{Section 1}

Report goes here.

\section{Introduction}

Feel free to add more sections/subsections as necessary.

\section{Methodology}

Here is a way to add tables:
\begin{table}[H]
    \centering
    \begin{tabular}{lll}
               & Test Column 1 & Test Column 2 \\
    Test Row 1 & $1$           & $2$           \\
    Test Row 2 & $3$           & $4$          
    \end{tabular}
\end{table}
You can use this \href{https://www.tablesgenerator.com/}{website} to generate tables.

To add figures, you can read this \href{https://www.overleaf.com/learn/latex/Inserting_Images}{short guide} by Overleaf.

To add inline equations, you can do $f(x) = y$. To add line-broken equations, you can do
\[f(x) = y.\]
To add numbered equations, you can do
\begin{equation}
    f(x) = y.
\end{equation}
To add multiline equations, you can look up the \texttt{align} environment (and its unnumbered variant \texttt{align*}).

\section{Results}

\printbibliography

\end{document}
